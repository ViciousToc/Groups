\documentclass[a4paper]{article}

%% Language and font encodings
\usepackage[english]{babel}
\usepackage[utf8x]{inputenc}
\usepackage[T1]{fontenc}

%% Sets page size and margins
\usepackage[a4paper,top=3cm,bottom=2cm,left=3cm,right=3cm,marginparwidth=1.75cm]{geometry}

%% Useful packages
\usepackage{amsmath}
\usepackage{amsfonts}
\usepackage{graphicx}
\usepackage[colorinlistoftodos]{todonotes}
\usepackage[colorlinks=true, allcolors=blue]{hyperref}

\title{Judson's Abstract Algebra: Chapter 7}
\date{}

\begin{document}
\maketitle


\section*{1}

Encode IXLOVEXMATH using the cryptosystem in Example 66.

\vspace{\baselineskip}

LAORYHAPDWK


\section*{2}

Decode ZLOOA WKLVA EHARQ WKHA ILQDO, which was encoded using the cryptosystem in Example 66.

\vspace{\baselineskip}

WILLXTHISXBEXONTHEXFINAL


\section*{4}

What is the total number of possible monoalphabetic cryptosystems? How secure are such systems?

\vspace{\baselineskip}

There are $26! - 1$ possible systems (we subtract one because of the identity permutation). They are relatively insecure because of frequency analysis and other methods.


\section*{5}

Prove that a  $2 \times 2$ matrix $A$ with entries in $\mathbb{Z}_{26}$ is invertible if and only if $\gcd(\det(A), 26) = 1$.

\vspace{\baselineskip}

Assume that $A$ is invertible. This implies that there exists $A^{-1}$ such that $AA^{-1} = I$ which in turn implies

$$\det(A) \det(A^{-1}) = 1.$$

This shows that $\det(A) \in U(26)$. It is true that all $x \in U(26)$ are coprime to 26 and hence 

$$\gcd(\det(A), 26) = 1.$$

Assume that $\gcd(\det(A), 26) = 1$. There exists integers $r$ and $s$ such that

\begin{align*}
\gcd(\det(A), 26) &= \gcd(ad - bc, 26) \\
&= r(ad-bc) + 26s \\
\end{align*}

Hence $1 = r(ad-bc) + 26s$ rearranging yields

$$r(ad - bc) = 1 - 26s.$$

Since $s$ is an integer the right hand side is non-zero and hence $ad - bc \neq 0$ which implies that $A$ is invertible.


\section*{6}

Given the matrix

$$A = \begin{pmatrix}
3 & 4 \\
2 & 3
\end{pmatrix},$$

use the encryption function $f(\mathbf{p}) = A \mathbf{p} + \mathbf{b}$ to encode the message CRYPTOLOGY, where $\mathbf{b} = (2,5)^t$. What is the decoding function?

\vspace{\baselineskip}

The ciphertext of the message is YIEULHOKKYPN. The decoding function is $f^{-1}(\mathbf{p}) = A^{-1} (\mathbf{p} - \mathbf{b})$. Where 

$$A^{-1} = \begin{pmatrix}
3 & 4 \\
2 & 3
\end{pmatrix}.$$


\section*{7}

Encrypt each of the following RSA messages $x$ so that $x$ is divided into blocks of integers of length 2; that is, if $x = 142528$, encode 14, 25, 28 separately.

$$n=3551, E=629, x=31$$

$$y = 31^{629} \mod 3551 = 2791$$

$$n=2257, E=47, x=23$$

$$y = 23^{47} \mod 2257 = 769$$


$$n=120979, E=13251, x=142371$$

$$y_1= 14^{13251} \mod 120979 = 112135$$
$$y_2= 23^{13251} \mod 120979 = 25032$$
$$y_3= 171^{13251} \mod 120979 = 442$$

$$n=45629, E=781, x=231561$$

$$y_1 = 23^{781} \mod 45629 = 4438$$
$$y_2 = 15^{781} \mod 45629 = 16332$$
$$y_3 = 61^{781} \mod 45629 = 31594$$









\end{document}